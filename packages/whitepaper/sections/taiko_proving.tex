\subsection{Proving Blocks} \label{sec:proving}

A proof needs to be submitted to Ethereum so that a block can be on-chain verified. We stress again that all proposed blocks are verified immediately because proposed blocks are deterministic and cannot be reverted. The prover has \emph{no} impact on the post-block state. The proof is only required to prove to the \underline{TaikoL1} smart contract that the L2 state transitions and the rollup protocol rules are fully constrained. These on-chain verified L2 states are made accessible to other smart contracts (and indirectly to other L2s) so they can have access to the full L2 state, which is critical for e.g. bridges (see Section \ref{sec:bridges}).

Blocks can be proven in parallel and so proofs may be submitted out-of-order. As a result, when proofs are submitted for blocks where the parent block is not yet verified, we cannot know if the proof is for the correct state transition. A proof on its own can only verify that the state transition from one state to another state is done correctly, not that the initial state is the correct one. As such, proving a block can create a Fork Choice which is an attestation that the block in question transits from a prover-selected parent block to a correctly calculated new world state. It is important to note that there is only a single valid fork choice per block: the fork choice that transitions from the last on-chain verified block to the next \emph{valid} proposed block. All other fork choices use an incorrect pre-block state.

A Fork Choice is a tuple of 3 elements:

\begin{eqnarray}
E \equiv (H_p, H_h, [(a_1, p^{z}_1, [p^{m_1}_1,...]),...])
\end{eqnarray}

where $H_p$ is the block's parent hash, $H_h \equiv \texttt{KEC}(\texttt{RLP}(H))$ is the hash of the proposed block, and $(a_i, p^{z}_i, [p^{m_1}_i,...])$ are the $i$-th prover's address and the proofs. $p^{z}$ is a proof that shows the state transition from the parent hash to the block hash is correct, and [$p^{m_1}$,...] are Merkle proofs in the storage, transaction, and/or receipt trie that prove the anchor transaction has been executed successfully as the first transaction of the L2 block.

Taiko accepts up to $K_{\mathrm{MaxProofsPerForkChoice}}$ proofs per fork choice. Proofs for the correct fork choice will be eligible for compensation. No limit is set on the number of fork choices as the protocol does not know which fork choice for a block is the correct one until the parent block is on-chain verified.

\subsubsection{Invalid Blocks} \label{sec:proving-invalid}

If a block fails to pass the Intrinsic Validity Function $V^l$, the block can be proven to be invalid using a valid throw-away L2 block $\dot{B}$ whose first transaction is an \texttt{invalidateBlock} transaction on the \underline{TaikoL2} smart contract with the target block's txList as the sole input. \texttt{invalidateBlock} will emit an \texttt{BlockInvalidated} event with the target block's txList hash as a topic. On L1, we only need to verify that:

\begin{enumerate}
\item The throw-away block $\dot{B}$ is valid, and;
\item The first event emitted in the block is a \texttt{BlockInvalidated} event with the expected txList hash. 
\end{enumerate}

The Fork Choice for an invalid block is:

\begin{eqnarray}
E \equiv (H_p, H_h, [(a_1, p^{z}_1, p^{m}_1),...]) \\
H_h \equiv K_{\mathrm{BlockDeadEndHash}}
\end{eqnarray}

Where $K_{\mathrm{BlockDeadEndHash}}$ is a special value marking this Fork Choice is for an invalid block; $p^{z}$ and  $p^{m}$ prove the throw-away block is invalid, not the target proposed block.

It's important to note that these throw-away blocks are never a part of the Taiko chain. The only purpose of the block is to be able reuse the EVM proving subsystem so that we can create proofs for blocks with unexpected transaction data.
