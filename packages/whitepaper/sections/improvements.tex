

\section{Future Improvements}\label{sec:improvements}

\subsection{Ethereum Data Blobs}\label{sec:datablobs}
EIP-4844 \cite{eip4844} (or similar) on Ethereum will, once enabled, allow data to be stored on L1 in a more efficient manner. Instead of storing the txList data in the L1 transaction data we will instead be able to store the data in a data blob. This data will be read directly from the KZG commitment in the ZK-EVM circuits without ever needing to access the data in an L1 smart contract.

\subsection{Block Validity Verification at Proposal Time}\label{sec:propose-proof}
Currently we accept blocks at proposal time even if the transaction data is invalid. Afterwards, we depend on provers to generate a proof that shows the block is invalid (see Section \ref{sec:proving-invalid}). We do this because the work required to verify all requirements imposed on the transaction data is expensive to verify on L1. Instead, we can require a proof together with the proposed block attesting that the block data is valid. This requires computing a proof, and so the requirement for this improvement is that this proof can be generated efficiently enough so that it is not a potential bottleneck for proposing blocks. Because verifying a proof is still quite expensive, this proof should not be verified immediately at block proposal time but should be verified as part of the block proof.

\subsection{Signature Compression}\label{sec:signature-opt}
Signatures can be removed from the block data as long as the proposer can show that all transactions in the proposed block have valid signatures. This can be achieved with the help of an accompanying proof when a block is proposed. As such, the burden of having to verify the signatures is shifted solely to the block proposer, so it needs to be possible to generate this proof efficiently. The block prover can then simply assume all transactions are valid and so there is no need for the prover to know the signatures. Note that this could have a very small impact on the transaction trie of a block as the signature data is not part of the transaction data anymore. If we want to keep the transaction trie the same with the signatures included the transaction trie will also have to be built by the block proposer.

\subsection{Block Data Compression}\label{sec:compression}
A big part of the cost of a rollup block is the data that is required to be stored on L1. It has been shown that standard general compression schemes like DEFLATE \cite{deflate} work well on transaction data. It is possible to implement these schemes efficiently in a circuit and so the data published on L1 can be compressed while the circuits can decompress the data again. This will make it possible to reduce the amount of data that needs to be published on L1, significantly reducing costs.

\subsection{Batched Proof Verification}\label{sec:proof-opt}
Verifying a proof on L1 is quite expensive. Instead of verifying each proof for each block separately we instead let block provers submit their proof for a block to L1 without the protocol immediately verifying it. Other provers can batch verify one or more of these block proofs in another proof which can then be submitted and verified on L1. This significantly reduces the proof verification gas cost in exchange of the cost of generating this extra proof and an extra delay in on-chain finalization. Note that there is no need for the protocol to impose any limitations on the number or the range of block proofs being verified. Any number of blocks at any positions in the chain are allowed to be batch verified. The proving fee system should automatically steer provers towards a system that is the most efficient while not significantly increasing the on-chain finalization time.

\subsection{Rate Limiting using EIP-1559}\label{sec:eip1559-propose}
Although rollups can have significantly higher network capacity than L1s, this capacity is not without limit. As such the protocol needs to be able to limit how much work the L2 network needs to do to keep up with the tip of the chain. Ethereum already has a mechanism in place to do just that with \cite{eip1559} that we can use as well. 

At block proposal we keep track of how much work (measured in gas) is required to process the block, while subtracting the amount of work the Taiko network can handle. This effectively creates a market for network capacity (in gas) per ETH. This will impact how expensive Taiko block space is (paid by the block proposer), the higher the demand the higher the network fee (a fee paid to the Taiko DAO). This way, rate limiting is achieved in a way that does not simply impose a hard and inefficient cap on the network, instead this mechanism allows users to utilize the network in a fair way while allowing the Taiko network to capture the created value. And because the same mechanism is used on Ethereum it allows Taiko to be Ethereum-equivalent (with some small implementation detail changes) even for this part of its network, which is not obviously the case for L2s.

\subsection{EIP-1559 Powered Prover fees}\label{sec:eip1559-proof}
Proving blocks requires significant compute power to calculate the proof to submit and verify the proof on Ethereum. Provers need to be compensated for this work as the network needs to attract provers that are willing to do this work. How much to pay for a proof is not obvious however:
\begin{enumerate}
\item The Ethereum gas cost to publish/verify a proof on Ethereum is unpredictable.
\item The proof generation cost does not necessarily match perfectly with the gas cost.
\item The proof generation cost keeps changing as proving software is optimized and the hardware used gets faster and cheaper.
\item The proof generation cost depends on how fast a proof needs to be generated.
\end{enumerate}

Because the proving cost impacts the transaction fees paid by the users, the goal is to pay only as much as required for the network to function well. This means not underpaying provers because blocks may remain unproven, but certainly also not overpaying provers so that it doesn't make sense to incur very high costs to try and generate proofs as quickly as absolutely possible. A good balance is key to a well working solution that takes into account the needs of the different network participants.

It's clear that a fixed proving fee does not work. The protocol should also not be dependent on a single prover for a block because this will put too much power in the hands of a single entity that can impact the stable progress of the chain. 

It can be observed that this problem is very similar to the rate limiting problem described in Section \ref{sec:eip1559-propose}. The network, somehow, has to find the correct price between two resources where the demand/supply is ever changing. We can model this problem as a market between the proving fee (per gas) per proof delay (per time unit), striking a dynamic balance between proving cost and proof delay.

An additional complication is that the protocol expects the block proposer to pay the proving fee at block proposal time. As such, the \emph{baseFee} of this model is used to charge the proposer of a block using the total gas used in the block. This is only an estimate of the actual cost because the actual cost is only known when the proof is submitted. If the estimated cost was too high the difference is returned to the block proposer and the \emph{baseFee} is decreased. If the estimated cost was too low extra Taiko tokens are minted to make up the difference and the \emph{baseFee} is increased. To lower the chance that the estimated cost is too low and extra Taiko tokens need to be minted, a slightly higher \emph{baseFee} can be charged to the proposer than the one predicted by the model.

\subsection{Leverage Staking Withdrawal Support for the Ether Bridge}\label{sec:withdrawals}
Once withdrawing staked Ether is supported by Ethereum we will be able to use the same infrastructure to bridge Ether. Although this is still a work in progress and the final spec is still unknown, this should provide a more standard solution than the system described in Section \ref{eth-bridge}.
