\section{Previous Work}

The Ethereum ecosystem began looking towards layer-2 solutions for scaling beginning in 2017 with Plasma\cite{plasma}. Layer-2s move computation off-chain, and keep data either on Ethereum, or also off-chain.

Rollups, which put some compressed data per transaction on Ethereum, emerged as the leading scalability path for Ethereum over the past four years or so, drawing more interest and excitement versus other layer-2 solutions (Plasma and State Channels) due to the strong security guarantees they offer, as well as the broader range of applications they can support. Initially designed and proposed by Vitalik Buterin\cite{vitalik-rollup} and Barry Whitehat\cite{barry-rollup} and other Ethereum researchers in 2018, ZK-Rollups were implemented on Ethereum mainnet since 2019, beginning with Loopring. 

A drawback of ZK-Rollups back then was that due to constraints on ZKP capability, they were application-specific and not generalizable, thus precluding many Ethereum use cases and composability. The full power of the EVM could not be wielded within such an environment. A different form of rollup, Optimistic rollups, such as those implemented by Optimism and Arbitrum in 2021, were able to achieve EVM-compatibility, relying on cryptoeconomic games to verify state transitions with fraud proofs, as opposed to validity proofs. Among the drawbacks of relying on fraud proofs instead of validity proofs are 1) reliance on network participants to find incorrect state as opposed to reliance on cryptography, and 2) a relatively lengthy time to finality, which can delay moving assets out of the L2, as well as hinder cross-rollup composability. 

The holy grail was widely recognized to be the best of both worlds: EVM rollups, with computation verified by ZK proofs. These ZK-EVMs have been in the works for a few years, with projects such as zkSync, Starkware, Polygon, and Scroll building implementations, and the Ethereum Foundation playing a critical role in R\&D, with their Privacy and Scaling Explorations unit\cite{pse}. Advancements by other projects and researchers, such as ZCash and Aztec have also greatly advanced the ZK proving systems required. The differences in implementations mainly exist in how closely the rollups will support the EVM, versus make adjustments towards a ZK-favourable VM. The primary trade-off today is thus between EVM-compatibility, and ZK-efficiency for proof generation. Taiko’s aim is to prioritize EVM-equivalence down to the opcode level, and Ethereum-equivalence at the broader systems level, while mitigating any proving performance drawbacks via protocol design, which we describe in the rest of this paper. 

